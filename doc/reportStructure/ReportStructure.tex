\documentclass[11pt, a4paper]{report} 
\usepackage[utf8]{inputenc}
\usepackage[english]{babel}
\usepackage{textcomp,amsmath,longtable}
\usepackage{latexsym}


\begin{document}
	\chapter{Introduction}
	Introduksjonsdel som består av intro, metode og studie. Det står chapter 1, men tenk på det som parts og underkaptilene som kaptiler, bare gjorde det sånn så det skulle være lettere å ha oversikt. Noen av titlene er work in progress ennå.
	
	\section{Introduction}
	Dette kapittelet dreier seg om introduksjon og sånn. Bakgrunn, motivasjon, problemstilling, struktur. Referere til prosjektoppgave med kjell osv.
	
	\section{Methodology}
	Dette kapittelet dreier seg om metode. Tenkte å ha med noen research questions og sånn for å sette grunnlaget til rapporten. Snakke om hva som skal utføres.
	
	En utfordring: Begge spillerene har forskjellige roller, men begge er aktive.
	
	Hvordan fordele arbeidsoppgaver på en lur måte.
	
	Er dette noe som er engasjerende for barn, med samarbeidsspill.
	
	Er det noen forskjeller mellom gutt/jente, spillbakgrunn?.
	
	Se om det er flere ting fra spøøreundersøkelsen 
	
	Ha et spørsmål ang teknologi.
	
	Bruke motion teknologi fra konsoller til pc.
	
	\section{Study}
	Ha med collaborative gaming, viktig! EN slags oppsumering av games history. Mest mulig av immersion. Oppsumering av interaction devices. Det som er relevant for oppgaven.
	 
	Greit å ta ting rett fra prosjektoppgaven, men nevne.
	Dette kapittelet tenkte jeg at skal være en blanding av literatur studie med gameflow og sånn, og bakgrunnsstudie av diverse teknologier som ble vurdert osv (move, wii, kinect, unreal, unity etc.). Her kommer prosjektoppgaven inn.
	
	
	\chapter{Results}
	Denne delen er ment som resultatdelen og beskriver softwaren som er utviklet og testingen.
	
	\section{Game}
	Ha en egen del til Game. Concept, user requirements, implementation, testing underveis - sjøl og med feedback fra veileder. kreativ utvikling. utfordringer, med teknologi og move osv. Move, med plugin og api.
	Her vil jeg snakke om hvordan utviklingen av spillet foregikk og eventuelt hvilke utfordringer jeg støtte på underveis. Gir det mening å gå i dypere detalj og kanskje nevne litt om struktur, arkitektur, patterns og klassediagram og div? Kanskje dele dette i to kapitler, ett om utviklingsprosessen, og et om selve softwaren som forklarer gameplay osv.? Vet ikke om prosessen er så interessant når jeg er bare en person?
	
	\section{Experimentation}
	basert på forskningsspørsmålene og gameflow.
	forbredelser til testen, finne målgruppe, ta kontakt osv.
	Gjennomfæring
	Det var vi som skrev svarene og forklarte spørsmålene.
	Lengden på spørreskjemaet var passe, kunne ikke vært lengre, noen mistet oppmerksomheten.
	Klasserom med praktiske ting.
	Ha med bilder.
	Dette kapitlet beskriver brukertestingen som (forhåpentligvis) ble utført inkludert spørreskjema, beskrivelse av hvordan den ble laget og resultatene.
	Resultat.
	husk å nummerere spørreskjemaene.
	Evaluere resultatene underveis, tenke på de diverse gameflow.
	
	merke til:
	Viktig å forklare spørsmålene bra.
	En av de beste svarte litt motsatt av hva man skulle tru konsentrerte seg lite lns.
	Kanskje litt vannskelig, men 
	De ville bytte ofte på den første, men etterhvert så ble det litt å gjøre for hånda.
	Det var mye snakking, nesten alle snakket sammen.
	De beste hadde klar måte å kommunisere på, kommando språk, innspill til hverandre.
	
	
	
	\chapter{Evaluation and conclusion}
	Dette er en avsluttende del som beskriver analyse, konklusjon, begrensninger, vidre arbeid osv.
	
	\section{Evaluation}
	Analyse av testresultatene i dette kapitlet, trekke slutninger og div om resultatene. 	Skrive litt om begrensninger.
	
	\section{Conclusion}
	Trekke slutninger, og oppsumere, svare på research questions.
	
	\section{Further work}
	Vidre arbeid osv.
\end{document}