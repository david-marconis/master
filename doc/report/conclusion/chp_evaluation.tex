\chapter{Evaluation}
This chapter will evaluate the research done in this project, as well as an evaluation of the game prototype, an evaluation of the technology used, and an evaluation of the project as a whole.


\section{Evaluation of Research and Results}
The research goals will be revisited and the research questions will be answered through information gathered through the literature study, the experience from developing the game prototype and the results gathered from the play testing and questionnaire.

\subsection{Research Goal 1: Collaboration}
The first research goal was all about the collaboration and social interaction between players and was defined with the accompanied research questions as:
\begin{description}
	\item[RG1:] The purpose of this project is to create a collaborative game prototype which promotes collaboration and social interaction from the point of view of a game designer in the context of video game design.
	\item[RQ1.1:] What game mechanics are suitable for stimulating collaboration between players?
	\item[RQ1.2:] How can the game be designed to encourage social interaction between players?
\end{description}
From the game prototype (details in Chapter \ref{chp:design}) and from the experiment (details in Chapter \ref{chp:play_test}) it was clear that having the two players play different roles with different functions had a big impact on the collaboration between the players. With the players having different functions a dependency was created between the players as they could not do everything  by themselves. The function of having the character being able to pick up items that the hand could then place in the world created a bond between the players where they could help each other out. Dangerous elements like spikes and enemies that threatened the character were another way to create a bond between the character and the hand - the hand could protect the character from danger by blocking spikes with items and luring the enemies away with bait. This created more social interaction between the players where they would request or offer help to the other player. The puzzle aspect of the game also encouraged social interaction, as the players would think out loud and give directions to each other in order to solve the puzzles. A particularly interesting game element were the different switches that controlled coloured boxes and moving platforms. The players would try out the different switches, which was one of the key elements to solving the puzzles. The players would often discuss which switches were the correct ones to activate to solve the puzzle.

\subsection{Research Goal 2: Children's Experience}
The second research goal focused on the children's experience with the game prototype and how they would react to a collaborative game - it was defined with two research questions as follows:
\begin{description}
		\item[RG2:] The purpose of this project is to see how children experience a collaborative game where the focus is on collaboration and social interaction from point of view of the children in the context of playing video games.
		\item[RQ2.1:] Is a collaborative game that encourages social interaction engaging for children?
		\item[RQ2.2:] How does the children's gender or game experience affect their experience of collaborative game with focus on collaboration and social interaction?
\end{description}
The results from the questionnaire were analysed in Chapter \ref{chp:analysis}, and shows that the game prototype were engaging for the children that participated in the play testing. Everyone saw the importance in collaboration and communication for progressing in the game. The questionnaire was created with Sweetser's \cite{sweetser2005gameflow} GameFlow model in mind for better review of the game prototype. The results from the GameFlow analysis shows that the game prototype was good at capturing the concentration of the children, the feedback from the game was good, the children got easily immersed in the game, and the social interaction was good. The GameFlow analysis also uncovered that the game prototype's areas for improvement were in goal clarity, challenge, and autonomy.

With regards to the gender and game experience of the children, a few differences were noticed. The males found the game to be less challenging and showed a greater understanding of the game compared to the females by understanding the goal of the game more easily, and recognizing when they did things correct or wrong in the game. The male participants also seemed to get more immersed in the game by experiencing time going by fast, and found the game less challenging to a greater degree than the female participants. The males were also more competitively inclined despite the collaborative nature of the game. There were similar results for the children's prior game experience. Children with more game experience found it easier to understand the goal of the game, were more competitively inclined, and got more immersed in the game. However the participants with only some prior game experience found the game to be more adequately challenging.

\subsection{Research Goal 3: Game Development}
The third and final research goal was intended to explore the various technologies available for an independent video game developer, described with research questions as:
\begin{description}
	\item[RG3:] The purpose of this study is to investigate what tools are available to create video games from the point of view of an independent developer in the context of video game design. 
	\item[RQ3.1] What tools are available for developing video games as an independent developer.
	\item[RQ3.2] Can proprietary controllers made for game consoles be used to develop games for the PC?
\end{description}
There are many technologies that support the independent video game developer in creating video games, as discribed in Chapter \ref{chp:technology}, with the most important being the game engine. Game engines lessen the burden of the game developer by handling common video game functionality like rendering, physics, lighting, and networking. Two of the most popular game engines are Unreal Engine 4 and Unity3D 5 - these game engines have big online communities that can provide help and information. The entry level for these game engines are free so any aspiring game developer can test their skills in video game development without monetary risk. There are also free software tools available like \gls{ide}s for managing and writing source code, and version control systems for keeping track of the games history and for preventing loss and corruption of code. Game assets such as sound and graphics are also available online with both free and paid options, there are even possibilities for hiring a graphic or sound designer online. The assets for the game prototype in this project come from free online resources, details in Section \ref{sec:chosen_tech}.

From the experience of developing the game prototype, using proprietary controllers made for consoles on the PC is not a straightforward process. With some of the controllers not providing official APIs it can be difficult to get the controllers to work in the desired environment. This was experienced while trying to get the PlayStation Move controller to work with Unreal Engine 4, more details in Section \ref{sec:dev_limitations}. The Move controller was abandoned because of complications when integrating it in the game prototype.
	



\section{Evaluation of the Game Prototype}
\label{sec:eval_game}
The game prototype was a success at stimulating collaboration and encouraging social interaction and all of the children enjoyed the gameplay. The idea of the players having two different roles was well received by the children and created a unique dynamic that that made the players depend on each other. The other elements were also good for enhancing the gameplay and making it a game that the children easily got immersed in. 

However, there are also some room for improvement as made apparent by the results of the questionnaire. The challenge of the game was not consistent and was a little too difficult for the children. The goal of the game was not that apparent either. The game should have had some more introductory levels to teach the players the basics, which could have helped teach the players the mechanics of the game better and made the goal more apparent. There could also have been a story to the game, but this was omitted due to time constraints. The level design could also have been a little better and more adjusted to the skill level of the children, especially the first level. There was a spot in the first level where the players would get stuck and had a hard time getting out of.

The game is very open and could support a multitude of new gameplay elements. There are some enhancements that could have been made to make the hand more active like having the hand being in danger at times and the character had to rescue them - the way it is now the stress level for the hand is low compared to the character. For example there could have been some ghosts chasing the hand, and only the character could scare them away. There are currently also only two different items that the hand can grab and place in the world, other possible ideas for items include an anvil that can be dropped on enemies to kill them, ladders, suspension bridges, or ropes for the character to get around the world easier, or maybe a vehicle for the player to ride or fly in to cross big gaps or float across lava.

Children was the primary target audience for the game, but the gameplay elements can be combined to create more difficult levels that are challenging even for experienced adult gamers by leaving less room for error and punishing mistakes more. The graphics and audio could also be changed to change its look and feel. For example the game can be made more gloomy and dark to appeal to an older audience that enjoys horror style video games.

	
\section{Evaluation of Technology}
Unreal Engine 4 was decently easy to get into in the beginning. The gameplay framework was logical, and having the C++ code classes inherit from classes in the framework made it easy to start creating gameplay elements. The only problem was that a big part of the online tutorials for Unreal Engine 4 included or was entirely based on Blueprints. Blueprints a visual scripting language that can be used together with or in place of C++ classes. In the beginning, Blueprints seemed like a useless addition for an experienced programmer, as everything could be done using C++, and was considered a tool for inexperienced programmers. Even though Blueprints are useful for inexperienced programmers it can also be used by experienced programmers to create quick prototypes of features, and can also be used to change attributes of classes to create different variants.

There were some initial problems getting the PlayStation Move controller connected to the computer, but it was eventually achieved through an unofficial \gls{api}. A plugin for Unreal Engine that was based on this \gls{api} was discovered and tested working with the Move controller early on in the project and the technology seemed promising. However, when the time came to integrate the move controller to the game, there was no way of accession the raw positional data of the move controller on the screen, which made it hard to implement without much complications. More time should have been invested into trying to get the Move controller working, but since the plugin was tested and looked functional in the beginning of the project, it was assumed to provide the needed functionality.


\section{Evaluation of Project}
Working on a big project like this was a challenging and educational experience. To work on this project alone was both freeing in the sense of having the sole responsibility, but it was also a little frightening to face this challenge alone. The weekly development cycle was a good way of conducting the development because it broke down the project into intermediate chunks that could be completed on a week by week basis. The weekly meeting with the advisor was invaluable for feedback and direction on what to do next, and for keeping up the motivation for the project.

To work with children was a new and enlightening experience that posed some challenges. It was hard to judge the skill level of the children, which in turn made it hard to create appropriately challenging levels. The levels were made with an increasing difficulty, but it could have started a little lower in the challenge spectrum. It was impressing to see how well the children were able to communicate and work together to solve the levels, and they were all genuinely excited to be able to play. All of the children were well behaved and no problems occurred during testing, other that technical issues. The children would also come with unexpected feedback - some of the children asked the name of the game and also the name of the character, or what planet they were on. One child said that they were competing with the spider when asked if they felt that they competed with the other players. Thoughts and statements like that made the experience more fun and rewarding, and provoked some interesting new perspectives. 
%\note{Skrive litt om hvordan det var å jobbe med et stort prosjekt aleine. Hvordan veildeningen gikk og påvirket arbeidet. Hva jeg har lært fra prosjektet, og litt om hvordan det var å jobbe med/utvikle spill for barn.}
%Hvordan var det å jobbe aleine
%alf inge hjalp ganske bra
%Hva har jeg lært fra dette
%hvordan var det å jobbe med barn
