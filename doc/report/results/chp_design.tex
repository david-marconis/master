\chapter{Design}
This chapter covers the design of the video game prototype that was created. It will provide insight into the concept of the game, the requirements, and the gameplay elements of which the game is comprised.


\section{Game Concept}
\subsection{Original Concept}
The original concept of the game was created in the specialization project with the idea of an action platformer with two players that have different roles. Player one controls a game character using a traditional gamepad, while player two controls the game world using a motion control device. The players have to cooperate and help each other to progress through the game and complete the different levels of the game. Figure \ref{fig:original_concept} shows the original game concept with one player using a gamepad to control a character and the other player using hand gestures.

\hfigure{original_concept}{Original concept: One player lowers a bridge using hand gestures to help the other player cross the water}{1}


\subsection{Final Prototype}
The core concept of the game is cooperation; neither of the players should be able to complete a level without the cooperation of the other player. Both players should be equally engaged, but have different roles and tasks during the game. The original concept experienced an evolution throughout the development period, detailed in section \ref{sec:evolution}, and the final concept of the game prototype took shape. 

The game is a puzzle platformer that has different levels where the players have to cooperate to reach the goal. Player one has to reach the goal of each level, but cannot do so because there are obstacles in the way that can only be manipulated by player two. In order to unlock the goal the character needs to collect a certain number of coins on that level. To be able to reach some places, player two needs to place blocks that the character can stand on, creating bridges and towers. There are enemies in the game that will chase the character and hurt them, but the players cannot hurt the enemies. The player controlling the hand can lure enemies away by using bait, allowing the player to access areas guarded by enemies. Player two can also activate switches that can control moving platforms, or make coloured boxes appear and disappear. Figure \ref{fig:level1} shows a part of the first level where the hand must activate coloured switches in order to remove the corresponding coloured boxes that block the players movement.
\note{Er dette for detaljert til å være concept?}

\hfigure{level1}{A puzzle with coloured switches that control the existence of corresponding boxes}{1}




\section{Requirements}
Because the game prototype is a creative initiative there were no pre determined requirements. The requirements presented here are meant to make sure that the prototype adheres to the core concepts of the game and allows for interesting gameplay and cooperation between players. Table \ref{tab:requirements} contains a list of the requirements for the game prototype. \note{Lista med requirements er ikke ferdig, bare lurer på om det ser greit ut. Er det sånn at jeg burde lage use cases/user stories/scenarios til requirementsa osv.?}



\begin{table}[!ht]
	\centering
	\caption{The requirements for the game prototype}
	\label{tab:requirements}
	\begin{tabular}{|l|l|}
		\hline
		\textbf{ID} & \textbf{Description}                                  \\ \hline
					& \textbf{Functional requirements}                      \\ \hline
		RQ1         & The game must support two players                     \\
		RQ2         & The two players must have different roles in the game \\
		RQ3         & The game must support multiple levels                 \\ \hline
					& \textbf{Usability requirements}                       \\ \hline
		RQ4         & The two players should be equally engaged             \\
		RQ5         & All levels should require both players involvement    \\
		RQ6         & All levels should be possible to complete             \\ \hline
	\end{tabular}
\end{table}

\section{Gameplay Elements}
\subsection{Character}
\subsection{Hand}
\subsection{Game World}
\subsection{Springboard}
\subsection{Enemies}
\subsection{Spikes}
\subsection{Items}
\subsection{Bait}
\subsection{Moving Platforms}
\subsection{Coloured Blocks}



