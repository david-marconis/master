\chapter{Design}
\label{chp:design}
This chapter covers the design of the video game prototype that was created. It will provide insight into the chosen technologies for the game, the concept of the game, the requirements, and the gameplay elements of which the game is comprised.

\section{Chosen Technologies}
\label{sec:chosen_tech}
This section describes the technologies that were chosen for the development of the game prototype.

\subsection{Interaction Devices}
The game will use a PlayStation Move motion controller and a gamepad to control the characters in the game. These interaction devices were chosen with the game concept in mind, more in Section \ref{sec:concept_final}, and supports a design where the players have different roles and will create a unique dynamic between the players. PlayStation Move controller will control the hand that can manipulate the game world, this controller is suitable for representing the players hand with precision and provides buttons for manipulating the environment. The gamepad will control the character that can interact with the game world, and provides a familiar interaction device for a platformer game. The players will have entirely different functions in the game and must use these functions to work together. The PlayStation Move was also chosen to see how a proprietary game controller for a console can be used in a PC game.

\subsection{Game Engine}
The game will be developed using the Unreal Engine 4 as the game engine. This ease the development by handling common video game tasks such as rendering, collision, and physics. Unreal Engine 4 has a plugin called Paper2D which helps in the creation of 2D games. There is also an unofficial plugin for the PlayStation Move controller to make it easy to use with games developed in Unreal Engine 4. Unreal Engine 4 is a popular game engine with a big community and good documentation online. It also uses C++ as a programming language that the author is familiar with, which is one of the main reasons that it was chosen over Unity3D 5 that has similar features to Unreal Engine 4.

\subsection{Software Tools}
Microsoft's Visual Studio 2015 will be used as the \gls{ide} for the development, as it is the default \gls{ide} for Unreal Engine 4. The project will also use Git as version control, arguably the most popular version control systems today. A version control system will help keep a track of the changes throughout the development, and serve as a safety net in case the files get lost or corrupted locally.

\subsection{Game Assets}
The game prototype will use free online game resources for graphics and audio. This is chosen to see if a game can be created using free resources, and because the game is supposed to be a prototype with limited monetary resources. The graphics used in the game prototype are created by Kenny Land \cite{kenney2016platformer} and are provided with a Creative Commons CC0 1.0 licence, which means that they are free to use for personal and commercial projects. The background music for the prototype is by TomCat Carty \cite{carty2015space} and is provided as freeware. Most of the sound effects come from Sonnis.com's Game Audio Bundle \cite{sonnis2016audio}, and some from JewelBeat.com \cite{jewelbeat2016audio}, both of which allow for use in personal or commercial products for free.

\subsection{Summary}
The game prototype will use a gamepad and the PlayStation Move controller to control the game. The game prototype will use the Unreal Engine 4 as the game engine for development. In addition to this, Microsoft Visual Studio 2015 will be used along with Git to aid with the programming, managing of the source code, and will keep a history and online backup of the source code. The game's graphics and audio will come from free online assets.


\section{Game Concept}
The main motivation behind the game concept is to have two players working together using two different controllers to navigate through various challenges. This is meant to stimulate collaboration between players.

\subsection{Initial Concept}
The initial concept of the game was created in the specialization project with the idea of an action platformer with two players that have different roles \cite{hovind2015alternative}. Player one controls a game character using a traditional gamepad, while player two controls the game world using a motion control device. The players have to cooperate and help each other to progress through the game and complete the different levels of the game. Figure \ref{fig:original_concept} shows the original game concept with one player using a gamepad to control a character and the other player using hand gestures.

\hfigure{original_concept}{Early concept sketch: One player lowers a bridge using hand gestures to help the other player cross the water}{1}


\subsection{Final Prototype}
\label{sec:concept_final}
The core concept of the game is cooperation; neither of the players should be able to complete a level without the cooperation of the other player. Both players should be equally engaged, but have different roles and tasks during the game. The original concept experienced an evolution throughout the development period, detailed in Section \ref{sec:evolution}, and the final concept of the game prototype took shape. 

The game is a puzzle platformer that has different levels where the players have to cooperate to reach the goal. Player one has to reach the goal of each level, but cannot do so because there are obstacles in the way that can only be manipulated by player two. In order to unlock the goal the character needs to collect a certain number of coins on that level. To be able to reach some places, player two needs to place blocks that the character can stand on, creating bridges and towers. There are enemies in the game that will chase the character and hurt them, but the players cannot hurt the enemies. The player controlling the hand can lure enemies away by using bait, allowing the player to access areas guarded by enemies. Player two can also activate switches that can control moving platforms, or make coloured boxes appear and disappear. Figure \ref{fig:level1} shows a part of the first level where the hand must activate coloured switches in order to remove the corresponding coloured boxes that block the players movement.

The game prototype has a total of five levels with varying degrees of difficulty and different puzzles to solve.

\hfigure{level1}{A puzzle with coloured switches that control the existence of corresponding boxes}{1}


\section{Requirements}
Because the game prototype is a creative initiative, there were no pre determined requirements. The requirements presented here are meant to make sure that the prototype adheres to the core concepts of the game and allows for interesting gameplay that encourages social interaction and stimulates collaboration between players. Table \ref{tab:requirements} contains a list of the requirements for the game prototype. These requirements will be helpful when creating and testing the prototype to ensure that they are met.

\begin{table}[!ht]
	\centering
	\caption{The requirements for the game prototype}
	\label{tab:requirements}
	\begin{tabular}{|l|l|}
		\hline
		\textbf{ID} & \textbf{Description}                                                  \\ \hline
		            & \textbf{Functional requirements}                                      \\ \hline
		RQ1         & The game must support two players                                     \\
		RQ2         & The two players must have different roles in the game                 \\
		RQ3         & Player one controls a character that can interact with the game world \\
		RQ4         & Player two contols a hand that can manipulate the game world          \\
		RQ5         & The character will be controlled by a gamepad                         \\
		RQ6         & The hand will be controlled by the PlayStation Move                   \\
		RQ7         & The game must support multiple levels                                 \\
		RQ8         & All levels must have a visible goal that leads to the next level      \\
		RQ9         & The player must collect coins in order to unlock the goal             \\ \hline
		            & \textbf{Usability requirements}                                       \\ \hline
		RQ10        & All levels should require both players involvement                    \\
		RQ11        & The two players should be equally engaged                             \\
		RQ12        & The game should have different challenging game obstacles             \\
		RQ13        & All levels should be possible to complete                             \\ \hline
	\end{tabular}
\end{table}

\section{Gameplay Elements}
\label{sec:gameplay_elements}
This section describes the different gameplay elements that make up the game prototype.

\subsection{Character}

\hfigure{character}{The visual representation of the character}{0.1}

The character is represented by an alien avatar, illustrated in Figure \ref{fig:character}, which can move around and interact with the game world. It is controlled by player one, and its only controls are left-right movement and jumping. The character may not harm enemies, nor interact with elements that can manipulate the world, like switches or bait. The primary objective of the character is to collect coins and reach the exit of the level while dodging enemies, pitfalls, and spikes to stay healthy. When the character loses all of its health, the character dies and the level is restarted.

\subsection{Hand}

\hfigure{hand}{The visual representation of the hand}{0.1}

Shown in Figure \ref{fig:hand} is the hand, which is a powerful entity that can manipulate the game world. The hand can move up, down, left, and right, and is not affected by gravity or obstacles in the game world like walls and enemies. By using switches and items, the hand can manipulate the game world in order to provide a safe passage for the character.

\subsection{Springboard}

\hfigure{springboard}{The springboard is ready to launch the player.}{0.1}

The springboard is pictured in Figure \ref{fig:springboard} and will launch the character high up in the air whenever it jumps on the springboard. This can be used to reach places that is otherwise to high for the player to reach. The spring board will only launch players and not enemies or items.

\subsection{Enemies}

\hfigure{enemies}{The snail(left) and the spider(right) will chase the character.}{0.3}

There are two different enemies in the game: the spider and the snail, both pictured in Figure \ref{fig:enemies}. The snail and the spider will both chase the character when it gets close to them, but the spider has a higher movement speed than the snail. If the enemies reach the character they will damage the character and cause it to be knocked back. The enemies cannot be killed by the character or hand directly, but they can be lured off a cliff. The spiders can be lured away by the hand by using bait, or lured off a cliff by making it follow the player off.

\subsection{Spikes}

\hfigure{spikes}{Spikes are harmful for the character to jump on}{0.1}

Spikes can be seen in Figure \ref{fig:spikes} and is harmful for the character. When the character hits the spikes from above, it will take damage and be knocked up in the air.

\subsection{Items}

\hfigure{item}{A grass building block item}{0.1}

Items are building blocks that can be picked up by the character and then placed in the world by the hand. Figure \ref{fig:item} shows a building block item. The items are found throughout a level and it must first be picked up by the character before the hand can place it in the world. When an item is picked up, it goes into an inventory, displayed in the top right of the screen, which the hand can then click on to grab the item to be placed. The character can not stand on the items unless they have been placed by the hand, and the hand may not pick up the item. This is done do create a dependency between the character and the hand. After the item has been placed it can be grabbed again by the hand in order to reduce the penalty for erroneous placement. Multiple items can be used together in order to create a bridge or a tower for the character to reach otherwise unreachable places.

\subsection{Bait}

\hfigure{bait}{Bait can be used to lure spiders}{0.1}

The bait, pictured in Figure \ref{fig:bait}, is used for luring spiders away from the character in order for the character to reach an area guarded by a spider. Bait works like items where the character picks the bait up from the world, the bait gets added to the inventory, and the hand can then place the bait out in the world. When a spider gets close to a bait, it will run to the bait and start eating it, ignoring the player. When the spider finishes eating the bait, it will continue chasing the character if it is close. The bait can be placed in the world by the hand to immobilize the spider for a short period, or the bait can be held by the hand to make the spider chase the hand instead of the player.

\subsection{Moving Platforms}

\hfigure{platform}{A moving platform with the switch attached}{0.3}

Moving platforms, shown in Figure \ref{fig:platform}, are platforms that can be set in motion or stopped by a switch. The character can stand on the platform and move along with it, but only the hand can start and stop the movement of the platform. The switch to turn the platform on or off can either be attached to the platform itself, or it could be stationary, attached to the floor or a wall.

\subsection{Coloured Blocks}

\hfigure{colour_block}{A basic opaque colour block}{0.1}

 Coloured blocks work similar to the moving platforms in that they are also controlled by switches. A basic colour block is shown in Figure \ref{fig:colour_block}, and a level with multiple colour blocks can be seen in Figure \ref{fig:level1}. The colour blocks can be activated and deactivated using a switch, which makes them appear or disappear in the game world. When a colour block is deactivated it becomes transparent and does not block the movement of the character. However, when the colour block is activated, it becomes opaque and blocks the character movement. Activating and deactivating the coloured blocks can create bridges, open doors, and trap enemies, but it can also make the player fall or even trap the player by blocking its movement.

\section{Summary}
The game prototype contains different gameplay elements that are meant to encourage cooperation between the players. In order to complete the levels, the players need to work together to solve the puzzles involved. The main objective of the character is to run and jump around while collecting coins, picking up items for the hand, and avoiding enemies in order to reach the goal. The character can not reach the goal though without the hand placing items, activating moving platforms and colour boxes, and luring enemies away with bait. Figure \ref{fig:level4} shows the character using items placed by the hand in order to reach a ledge and avoid an enemy.

\hfigure{level4}{The character and hand must work together to reach the ledge}{1}