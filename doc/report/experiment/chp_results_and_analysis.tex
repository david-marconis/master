\chapter{Results and Analysis}
This chapter presents the results of the play testing session along with observations that were made and analysis of the results.


\section{Results}
Table \ref{tab:simple_results} shows the agreement and disagreement of the questions that were asked. For simplicity's sake, agreement and strong agreement have been aggregated, likewise with disagreement and strong disagreement.
For more detailed information regarding the results, like statistical mean, variance and standard deviation of the questions from the questionnaire can be found in Appendix X. \note{insert appendix} The ages of the participants were ranging from six to nine years old with the composition being: six six year olds (32\%), five seven year olds (26\%), five eight year olds (26\%), and three nine year olds (16\%). Out of the total of 19 participants, 8 of them were female (44\%). Over half of the children told that they had a lot of prior game experience (58\%), seven said that they had some game experience (37\%), and one told that they had no prior game experience(5\%).

%TODO: Endre tabellen til å være mye enklere, ha detaljer i appendix

\begin{table}[]
	\centering
	\caption{The results from the questionnaire}
	\label{tab:simple_results}
	\begin{tabularx}{\textwidth}{|l|X|l|l|}
		\hline
		\textbf{ID} & \textbf{Question}                                              & \textbf{Agree} & \textbf{Disagree} \\ \hline
		\textbf{}   & \textbf{Concentration}                                         & \textbf{}      & \textbf{}         \\ \hline
		Q1          & I am concentrated when playing the game                        & 100,00\%       & 0,00\%            \\ \hline
		\textbf{}   & \textbf{Goal Clarity}                                          & \textbf{}      & \textbf{}         \\ \hline
		Q2          & It was easy to understand the goal of the game                 & 52,63\%        & 47,37\%           \\ \hline
		\textbf{}   & \textbf{Feedback}                                              & \textbf{}      & \textbf{}         \\ \hline
		Q3          & It is easy to understand when I do something wrong in the game & 78,95\%        & 21,05\%           \\ \hline
		Q4          & It is easy to understand when I do something good in the game  & 89,47\%        & 10,53\%           \\ \hline
		Q5          & I liked the graphics of the game                               & 100,00\%       & 0,00\%            \\ \hline
		Q6          & I liked the sounds of the game                                 & 78,95\%        & 21,05\%           \\ \hline
		\textbf{}   & \textbf{Challenge}                                             & \textbf{}      & \textbf{}         \\ \hline
		Q7          & I get bored when playing the game                              & 10,53\%        & 89,47\%           \\ \hline
		Q8          & The game was too easy                                          & 15,79\%        & 84,21\%           \\ \hline
		Q9          & The game was adequately difficult                              & 57,89\%        & 42,11\%           \\ \hline
		Q10         & The game was too hard                                          & 36,84\%        & 63,16\%           \\ \hline
		Q11         & My skill gradually improved when playing the game              & 89,47\%        & 10,53\%           \\ \hline
		\textbf{}   & \textbf{Autonomy}                                              & \textbf{}      & \textbf{}         \\ \hline
		Q12         & I felt in control of the movements of the player               & 68,42\%        & 31,58\%           \\ \hline
		Q13         & I could recover from errors I made in the game                 & 68,42\%        & 31,58\%           \\ \hline
		\textbf{}   & \textbf{Immersion}                                             & \textbf{}      & \textbf{}         \\ \hline
		Q14         & Time goes by fast when plaing the game                         & 84,21\%        & 15,79\%           \\ \hline
		Q15         & I become unaware of my surroundings when playing the game      & 89,47\%        & 10,53\%           \\ \hline
		\textbf{}   & \textbf{Social Interaction}                                    & \textbf{}      & \textbf{}         \\ \hline
		Q16         & I was competing with others to reach the furthest              & 31,58\%        & 68,42\%           \\ \hline
		Q17         & It was important to communicate when playing the game          & 100,00\%       & 0,00\%            \\ \hline
		Q18         & It was important to cooperate when playing the game            & 100,00\%       & 0,00\%            \\ \hline
		Q19         & I cooperated well with the other player                        & 89,47\%        & 10,53\%           \\ \hline
		Q20         & The other player cooperated well with me                       & 89,47\%        & 10,53\%           \\ \hline
		\textbf{}   & \textbf{Game Enjoyment}                                        & \textbf{}      & \textbf{}         \\ \hline
		Q21         & I enjoyed playing the game                                     & 100,00\%       & 0,00\%            \\ \hline
		Q22         & I would like to play this game at home                         & 100,00\%       & 0,00\%            \\ \hline
	\end{tabularx}
\end{table}


\section{Observations}
This section describes some general observations that were made during the play testing which were observed by the interviewers. The length of the questionnaire was an appropriate amount of questions, it could not have been any longer as some children seemed to lose their attention at the end. It was also important to explain the questions well, as some of them were not as easy to grasp for the children. Some questions also needed to be asked twice in order to get a sense of scale of agreement or disagreement. Questions like: "do you strongly agree or just slightly" were frequently asked when the detail was not provided in the initial response of the child. 

The majority of the children were very cooperative with each other and there were a lot of communication between them on things like how to conquer the different obstacles and what to do next. Learning the controls of the alien and the hand character was the focus of the first level, but the movement for the alien character was more complex, which led to some players wanting to switch controllers. However on the following levels the tasks for the hand character were more complicated and the players often wanted to switch back to the hand character. The game was possibly a little too hard for the children that we tested, as some were struggling to beat the first level.  Despite this, most of the children thought the game difficulty was adequate and all of the players thought that the game was fun and would like to play it at home.


\section{Analysis}
\note{Trenger fullføring}
\subsection{Concentration}
\subsection{Goal Clarity}
\subsection{Feedback}
\subsection{Challenge}
\subsection{Autonomy}
\subsection{Immersion}
\subsection{Social Interaction}
\subsection{Game Enjoyment}
\subsection{Background Differences}

\section{Summary}
